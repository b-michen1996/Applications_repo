%%%%%%%%%%%%%%%%% PREAMBLE %%%%%%%%%%%%%%%%%%%%%%%%%%%%
%Ändern Sie die Schriftgröße Ihres Dokuments – 10pt, 12.1pt, etc.
\documentclass[letterpaper,11pt,oneside, german]{article}
\usepackage[utf8]{inputenc}
\usepackage{setspace}
\usepackage[colorlinks, citecolor = blue, linkcolor = red, urlcolor = blue]{hyperref}
\usepackage{comment}
\usepackage{multirow}
\usepackage{makecell}
\usepackage{xcolor}
\usepackage[T1]{fontenc}
\usepackage{array}
\usepackage{tabularx,ragged2e,booktabs,babel}

\usepackage{graphicx}
\graphicspath{ {images/}} %laden Sie Ihre Unterschrift in diesen Ordner hoch
%Ändern Sie die Seitenränder passend zum Inhalt Ihres Lebenslaufs
\usepackage[left=1in, right=1in, bottom=1.25in, top=1.25in]{geometry}

%Skype-Informationen – geben Sie Ihren Skype-Namen für einen Link zur Kontaktaufnahme an
\newcommand*{\Skype}{\href{skype:john.smith?add}{john.smith}} 
\newcommand{\Absender}[1][\normalsize]{\Skype} 
\newcommand{\fscv}{\normalsize} 

%Ändert die Seitenzahlen – {arabic}=arabische Ziffern, {gobble}=keine Seitenzahlen, {roman}=Römische Ziffern
\pagenumbering{gobble}

%%%%%%%%%%%%%%%%% END OF PREAMBLE %%%%%%%%%%%%%%%%%%%%%
\def\CC{{C\nolinebreak[4]\hspace{-.05em}\raisebox{.4ex}{\tiny\bf ++}}}

\begin{document}
%%%%%%%%%%%%%%%%% NAME OF APPLICANT %%%%%%%%%%%%%%%%%%%

\noindent  \LARGE{\textbf{Lebenslauf}} \\
\noindent  \LARGE{\textbf{Benjamin Michen}} \\
\vspace{-2ex} 
\hrule 
\normalsize

%%%%%%%%%%%%%%%%% CONTACT INFORMATION %%%%%%%%%%%%%%%%%

\begin{center}
\begin{tabular}{r l r l}
Geburtsdatum: & 2.~September 1996 & \hspace{0.25in} E-Mail: & \href{mailto:benjamin.michen@gmx.de}{benjamin.michen@gmx.de} \\
Adresse: & Clara-Viebig-Str.~3, & \hspace{0.25in} Telefon: & 0178 8723598  \\
& 01159 Dresden, Deutschland 
\end{tabular}
\end{center}

%%%%%%%%%%%%%%%%% MAIN BODY %%%%%%%%%%%%%%%%%%%%%%%%%%%
\noindent \begin{tabular}{@{} l >{\raggedright\arraybackslash}p{0.85\textwidth} }
\fscv{Ausbildung} 
& \textbf{Abitur} mit Gesamtnote 1,4 am J.-G.-Herder-Gymnasium Pirna, 2013–2015. \\
&\\
& \textbf{B.Sc.\ Physik} mit Gesamtnote 1,6 an der TU Dresden, 2016–2019. \\
& Bachelorarbeit „Bulk Transport Properties of HgTe Nanostructures'' bei Prof.~Carsten Timm.\\
&\\
& \textbf{M.Sc.\ Physik} mit Gesamtnote 1,0 an der TU Dresden, 2019–2021. \\
& Masterarbeit „Disorder-Induced Exceptional Non-Hermitian Phases in Mesoscopic Quantum Systems'' bei Prof.~Jan Carl Budich.\\
& \\
& \textbf{Promotion Physik} an der TU Dresden, 2021–2025, bei Prof.~Jan Carl Budich. \\
& Dissertation „Nudging Quantum Matter over the Edge of Topological Order'', eingereicht im November 2025. Derzeit im Begutachtungsverfahren; Verteidigung voraussichtlich Anfang 2026. \\
& \\
\fscv{Forschungs-} & \textbf{Professur für Quanten-Vielteilchentheorie, TU Dresden, 2020--2025} \\
\fscv{erfahrung} & Mathematische und numerische Modellierung der Physik wechselwirkender Quantenvielteilchensysteme bei tiefen Temperaturen unter Prof.~Jan Carl Budich mit Schwerpunkten auf adiabatischer Zustandspräparation in Quantensimulatoren, Transporteffekten und topologischen Phasen. Siehe hierzu auch die Publikationsliste weiter unten. \\
&\\
& \textbf{Fraunhofer FEP Dresden, 2018--2021} \\
& Tätigkeit als studentischer Mitarbeiter für die Durchführung und Dokumentation von Elektronenstrahl-Experimenten sowie für numerische Simulationen elektromagnetischer Linsen mittels Finite-Elemente-Methoden. \\
&\\
\fscv{Lehr-} & \textbf{TU Dresden} \\
\fscv{erfahrung} & Übungsleiter für mehrere Lehrveranstaltungen der theoretischen Physik. \\
\\
\fscv{Sprachen und} & Deutsch (Muttersprache), Englisch (fließend), Spanisch (Niveau B1). \\
\fscv{Fähigkeiten} 
& Programmiererfahrung in Python, Julia, \CC, und Mathematica. \\
& Numerische Methoden für physikalisch motivierte Problemstellungen. \\
& Expertise in Quanteninformationsverarbeitung und Quantensimulation.\\
& Expertise im wissenschaftlichen Schreiben.\\
\\
\fscv{Weitere} & Work \& Travel in Neuseeland im Rahmen eines Gap Years, 2015–2016.\\
\fscv{Erfahrung} 

\end{tabular}

\clearpage

\renewcommand\refname{\LARGE Publikationsliste}


\begin{thebibliography}{Paper VII*}

\bibitem[Paper I]{Paper_I}
B. Michen, T. Micallo, and J.C. Budich, {\em Exceptional non-Hermitian phases in disordered quantum wires}, \href{https://journals.aps.org/prb/abstract/10.1103/PhysRevB.104.035413}{Phys. Rev. B {\bfseries{104}} (3), 035413 (2021)}.

\bibitem[Paper II]{Paper_II}
B. Michen and J.C. Budich, {\em Mesoscopic transport signatures of disorder-induced non-Hermitian phases}, \href{https://journals.aps.org/prresearch/abstract/10.1103/PhysRevResearch.4.023248}{Phys. Rev. Res. {\bfseries{4}} (2), 023248 (2022)}.

\bibitem[Paper III]{Paper_III}
B. Michen, C. Repellin, and J.C. Budich, {\em Adiabatic preparation of fractional Chern insulators from an effective thin-torus limit}, \href{https://journals.aps.org/prresearch/abstract/10.1103/PhysRevResearch.5.023100}{Phys. Rev. Res. {\bfseries{5}} (2), 023100 (2023)}.

\bibitem[Paper IV]{Paper_IV}
B. Michen and J.C. Budich, {\em Nonlinear interference challenging topological protection of chiral edge states}, \href{https://journals.aps.org/pra/abstract/10.1103/PhysRevA.108.L021501}{Phys. Rev. A {\bfseries{108}} (2), L021501 (2023)}.

\bibitem[Paper V]{Paper_V}
J. Schwardt, B. Michen, C. Lehmann, and J.C. Budich, {\em Exceptional Luttinger liquids from sublattice-dependent interaction}, \href{https://journals.aps.org/prb/abstract/10.1103/PhysRevB.110.245146}{Phys. Rev. B {\bfseries{110}} (24), 245146 (2024)}.

\bibitem[Paper VI]{Paper_VI}
B. Michen, T. Pokart, and J.C. Budich, {\em Adiabatic preparation of a number-conserving atomic Majorana phase}, \href{https://journals.aps.org/prb/abstract/10.1103/dqsb-nzx1}{Phys. Rev. B {\bfseries{112}} (4), 045112 (2025)}.

\bibitem[Paper VII]{Paper_VII}
B. Michen and J.C. Budich, {\em Quantum Hall Effect without Chern Bands}, \href{https://journals.aps.org/prl/abstract/10.1103/dgd8-4gzf}{Phys. Rev. Lett. {\bfseries{135}} (18), 186603 (2025)}.   

\end{thebibliography}

\end{document}
