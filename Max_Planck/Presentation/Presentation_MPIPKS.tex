\documentclass[ddcfooter, aspectratio=169, hyperref = {colorlinks,bookmarks=true,citecolor=blue,urlcolor=blue}]{beamer}
\usepackage[utf8]{inputenc}  % Kodierung der Datei
\usepackage[T1]{fontenc}  % Vollen Umfang der Schriftzeichen
\usepackage{amsmath}
\usepackage{amssymb}
%\usepackage{nameref}
\usepackage{isomath}
\usepackage{upgreek}
\usepackage{caption}
\usepackage{bbm}
\usepackage{mathtools}
\usepackage{mathrsfs}
\usepackage{amsmath}

\usepackage{graphicx}
\usepackage{wrapfig}

\usepackage{comment}

\setbeamercolor{bibliography entry author}{fg=black}
\setbeamercolor{bibliography entry title}{fg=black}
\setbeamercolor{bibliography entry location}{fg=black}
\setbeamercolor{bibliography entry note}{fg=black}

%\usepackage{hyperref}
%\usepackage[colorlinks,bookmarks=true,citecolor=blue,linkcolor=red,urlcolor=blue]{hyperref}


\usepackage{bm}
\usepackage{bbm}

\usepackage{subcaption}

%\usepackage{subfig}
%\usepackage{floatrow}

\newcommand\blfootnote[1]{%
  \begingroup
  \renewcommand\thefootnote{}\footnote{#1}%
  \addtocounter{footnote}{-1}%
  \endgroup
}


%sinnvolle Abstände im Inhaltsverzeichnis
\makeatletter
\patchcmd{\beamer@sectionintoc}
  {\vfill}
  {\vskip\itemsep}
  {}
  {}
\makeatother  

% Custom command to obtain the currently active label (of section, subsection etc.)
\makeatletter
\newcommand*{\currentname}{\@currentlabelname}
\makeatother

%Bilder neben Text
\newcommand{\lenitem}[2][.7\linewidth]{\parbox[t]{#1}{\strut #2\strut}}

\usetheme{Frankfurt}
\usecolortheme{beaver}
\setbeamertemplate{footline}[frame number]
\setbeamertemplate{navigation symbols}{}
\setbeamertemplate{headline}{}


\title[]{Previous research and future research plans}
\author{{\textbf{Benjamin Michen}}}
\date{07.01.2026}

\titlegraphic{\includegraphics[width=3cm]{Images/TUD_logo}%\hspace*{4.75cm}
%   \includegraphics[width=2cm]{Images/Grenoble_University.png}
}

% Insert slide with section name before each section
\AtBeginSection[]{
  \begin{frame}
  \vfill
  \centering
  \begin{beamercolorbox}[sep=8pt,center,shadow=true,rounded=true]{title}
    \usebeamerfont{title}\thesection~ \insertsectionhead\par%
  \end{beamercolorbox}
  \vfill
  \end{frame}
}




\begin{document}
\maketitle

\part{Previous Research}
\frame{\partpage}  % Creates a part title page

\section{Non-Hermitian physics from disorder and interactions}

\begin{frame}
\frametitle{\insertsectionhead}

\textbf{Effective non-Hermitian Hamiltonians}

\vspace{0.5em}
\pause
Take simple base Hamiltonian $H_0$ and add a perturbation $H_1$ (disorder, interaction...).

\vspace{0.5em}
\pause
$\Rightarrow$ Retarded single-particle Green's function acquires self-energy correction

\begin{alignat*}{2}
H_0 \;\Rightarrow&& \; G_0^R =&\; [\omega + i \eta - H_0]^{-1} \\
H_0 + H_1 \;\Rightarrow&& \;G^R =&\; [\omega - \underbrace{(H_0 + \Sigma)}_{H_\mathrm{eff}}]^{-1}
\end{alignat*}
\pause
Effective Hamiltonian usually non-Hermitian :
\begin{alignat*}{2}
H_\mathrm{eff}^\dagger \neq H_\mathrm{eff}
\end{alignat*}
\end{frame}

\begin{frame}
\frametitle{\insertsectionhead}

\textbf{Exceptional points}

\vspace{0.5em}

$H_\mathrm{eff}(k)$ will generically have non-diagonalizable degeneracies!\blfootnote{B. Michen, T. Micallo, and J.C. Budich,
\href{https://journals.aps.org/prb/abstract/10.1103/PhysRevB.104.035413}{Phys. Rev. B {\bfseries{104}} (3), 035413 (2021)}}

\vspace{1em}

\begin{figure}
  \begin{subfigure}{0.49\textwidth}  
  \includegraphics[width= \textwidth]{Illustrations/EP_spec.png}
  \subcaption*{Spectrum of disorder-induced $H_\mathrm{eff}(k)$.}
  \end{subfigure}
  \begin{subfigure}{0.49\textwidth}
  \includegraphics[width= \textwidth]{Illustrations/Transport_EP.png}
  \subcaption*{Transport signatures.}
  \end{subfigure}
\end{figure} 
\end{frame}

\begin{frame}
\frametitle{\insertsectionhead}
\textbf{Publications on exceptional non-Hermitian Physics}

{\color{black}
\begin{thebibliography}{Paper VII*}

\bibitem[]{Paper_I}
B. Michen, T. Micallo, and J.C. Budich, {\em Exceptional non-Hermitian phases in disordered quantum wires}, \href{https://journals.aps.org/prb/abstract/10.1103/PhysRevB.104.035413}{Phys. Rev. B {\bfseries{104}} (3), 035413 (2021)}.

\bibitem[]{Paper_II}
B. Michen and J.C. Budich, {\em Mesoscopic transport signatures of disorder-induced non-Hermitian phases}, \href{https://journals.aps.org/prresearch/abstract/10.1103/PhysRevResearch.4.023248}{Phys. Rev. Res. {\bfseries{4}} (2), 023248 (2022)}.

\bibitem[]{Paper_V}
J. Schwardt, B. Michen, C. Lehmann, and J.C. Budich, {\em Exceptional Luttinger liquids from sublattice-dependent interaction}, \href{https://journals.aps.org/prb/abstract/10.1103/PhysRevB.110.245146}{Phys. Rev. B {\bfseries{110}} (24), 245146 (2024)}.
\end{thebibliography}}

\end{frame}

\section{Adiabatic preparation of fractional Chern insulators from an effective thin-torus limit}

\begin{frame}
\frametitle{\insertsectionhead}
\textbf{\large The thin-torus limit of the fractional quantum Hall system \\}
\vspace{0.5em}
\pause
For FQH system on a torus of $L_x \times L_y$ send $L_x \to 0$ (while keeping total flux $B L_x L_y$ constant):
{\center $\Rightarrow$ Thin-torus (TT) limit of the FQH system \\}
\vspace{10 pt}

\pause
\begin{itemize}[<+->]
	\item System becomes exactly solvable
	\item TT ground states are charge-density waves, e.g., at filling $\nu = \frac{1}{2}$
	\begin{align}
	|GS_1\rangle = |101010...\rangle, \; \; 	|GS_2\rangle = |010101...\rangle  \nonumber
	\end{align}	
        \item The TT limit provides adiabatic connection between FQH state and CDW
\end{itemize}

\end{frame}

\begin{frame}
	\frametitle{\insertsectionhead}
	\textbf{ \large Continuously changing the aspect ratio of a lattice model\\}
	\begin{columns}[onlytextwidth,T]
		\begin{column}{.6\textwidth}
			\uncover<1->{
			Tune effective aspect ratio through the hopping ratio! 
			}
			\uncover<2->{
			\begin{align}
			\frac{L_y}{L_x} \propto \sqrt{\frac{|J_x|}{|J_y|}} \nonumber
			\end{align}
			}
			\uncover<3->{
			{\center $\Rightarrow$ Use this to adiabatically approach the TT limit of an FCI! \\}
			}
		\end{column}
		\begin{column}{.35\textwidth}
		\uncover<0->{
			\begin{figure}[htp!]	 
				\centering 
				\includegraphics[trim={0cm 0cm 0cm 0cm},clip, width=  \linewidth]{Images/Aspect_ratio.png} 
			\end{figure}
			}
		\end{column}
	\end{columns}
\end{frame}



\begin{frame}
	\frametitle{\insertsectionhead}
	\textbf{\large The coupled wire model}	\\

	\begin{columns}[onlytextwidth,T]
		\begin{column}{.5\textwidth}
\small
			\begin{align}
			H =& \sum_x \int_0^{N_y  l_B} \mathrm{d}y \left [ \Psi^\dagger_{x,y} \frac{\hat{p}_y^2}{2 m}\Psi_{x,y} +  \left( J e^{i \phi y}  \Psi^\dagger_{x,y} \Psi_{x +a,y} + \mathrm{H.c.} \right ) \right] \nonumber  \\
			&+ U \sum_{y} \int_0^{N_s l_B}  \mathrm{d}x \Psi^\dagger_{x,y} \Psi^\dagger_{x,y} \Psi_{x,y} \Psi_{x,y}, \nonumber 
			\end{align}
			
	\begin{itemize}[<+(1)->]
		\item Supports FCI state at filling $\nu = \frac{1}{2}$ of lowest band! 
		\item For large interwire hopping $J$, the FCI state becomes a CDW. Analytical solution possible!
	\end{itemize}
			
		\end{column}
		\begin{column}{.35\textwidth}
						\flushright
			\begin{figure}[htp!]	 
				\includegraphics[trim={-0.5cm -0.5cm 1.5cm 2.5cm}, width=  \linewidth]{Illustrations/coupled_wire_illustration.png} 
			\end{figure}
		\end{column}
	\end{columns}
	\blfootnote{B. Michen, C. Repellin, and J.C. Budich,
\href{https://journals.aps.org/prresearch/abstract/10.1103/PhysRevResearch.5.023100}{Phys. Rev. Res. {\bfseries{5}} (2), 023100 (2023)}}
\end{frame}

\begin{frame}
	\frametitle{\insertsectionhead}
	\begin{figure}[htp!]	 
			\centering 
			\includegraphics[trim={0cm 0cm 0cm 0cm}, width = 1 \linewidth]{Illustrations/ED_data_CW_complete.png} 
			\caption*{Left panel: ED data for energy gaps with analytical solution in green. Right panel: particle entanglement spectrum. \blfootnote{B. Michen, C. Repellin, and J.C. Budich,
\href{https://journals.aps.org/prresearch/abstract/10.1103/PhysRevResearch.5.023100}{Phys. Rev. Res. {\bfseries{5}} (2), 023100 (2023)}}}
	\end{figure}
\end{frame}

\section{Adiabatic preparation of a number-conserving Majorana phase}

\begin{frame}
\frametitle{\insertsectionhead}
\textbf{\large One-dimensional Majorana phase in number-conserving setting \\}
\vspace{0.5em}
Two quantum wires with pair-hopping interaction: \blfootnote{C. V. Kraus, M. Dalmonte, M. A. Baranov, A. M. Läuchli, and P. Zoller, \href{https://journals.aps.org/prl/abstract/10.1103/PhysRevLett.111.173004}{Phys. Rev. Lett. {\bfseries 111}, 173004 (2013)}.}
\begin{align}
H &= -t\sum_{\gamma = \mathrm{a,b}}\sum_{j = 1}^{L-1} \left[(c_{\gamma,j}^\dagger c_{\gamma,j+1} + c_{\gamma,j+1}^\dagger c_{\gamma,j}) \right] + W \sum_{j = 1}^{L-1} \left[c_{\mathrm{a},j}^\dagger c_{\mathrm{a},j+1}^\dagger c_{\mathrm{b},j} c_{\mathrm{b},j+1} + \text{H.c.} \right] \nonumber
\end{align}
\pause
\begin{itemize}[<+(1)->]
	\item Stabilizes two ground states with opposite wire parity $P_a = (-1)^{N_a}$
	\item $\mathbb Z_2$ symmetry-protected by $P_a $ 
	\item Tradeoff: Only works at critical point (gap closing $\propto 1/L$)\\
	$\Rightarrow$ Challenge for state preparation!
\end{itemize}

\end{frame}

\begin{frame}
\frametitle{\insertsectionhead}
\textbf{\large Strategy to prepare critical target state \\}
\vspace{0.5em}
Break symmetry $P_a = (-1)^{N_a}$ by interwire tunneling with flux $\phi$ to create energy gap 
\begin{align}
H_\phi &= r \sum_{j = 1}^L  \left[ e^{2 \pi i \phi j}c_{\mathrm{a},j}^\dagger c_{\mathrm{b},j} + e^{-2 \pi i \phi j}c_{\mathrm{b},j}^\dagger c_{\mathrm{a},j} \right]. \nonumber
\end{align}
\pause
$\Rightarrow$ Crucial: flux commensurate with filling $\phi = \nu$ (explained through bosonization, confirmed through DMRG)\blfootnote{B. Michen, T. Pokart, and J.C. Budich,
\href{https://journals.aps.org/prb/abstract/10.1103/dqsb-nzx1}{Phys. Rev. B {\bfseries{112}} (4), 045112 (2025)}}
\pause
\vspace{1em}

We start at finite value $r_0$ and ramp down to zero as power law:
\begin{align}
r_p(\tau) = r_{0} |1 -  \tau / \tau_\mathrm{tot}|^p, \quad \tau \in [0, \tau_\mathrm{tot}]. \nonumber
\end{align}
\pause
$\Rightarrow$ Optimal linear scaling of preparation time $\tau_\mathrm{tot}(L)$ by increasing $p(L)$ with system size $L$.
\end{frame}

\begin{comment}
\begin{frame}
	\frametitle{\insertsectionhead}
	\begin{columns}[onlytextwidth,T]
		\begin{column}{.45\textwidth}
			\textbf{\large Full protocol}	\\
			\vspace{0.5em}
			\begin{itemize}[<+->]
				\item Start from Mott state stabilized by staggered potential $\mu$
				\item Turn on $W$, $t$ and r
	                       \item Ramp down $r$ with power law in last stage ($p=1$ for illustration)
			\end{itemize}
		\end{column}
		\begin{column}{.45\textwidth}
						\flushright
			\begin{figure}[htp!]	 
				\includegraphics[trim={0cm 0cm 0cm 0cm}, width=  \linewidth]{Illustrations/gaps_protocol.pdf} 
				\caption{DMRG data for excitation gaps.}
			\end{figure}
		\end{column}
	\end{columns}
\blfootnote{B. Michen, T. Pokart, and J.C. Budich,
\href{https://journals.aps.org/prb/abstract/10.1103/dqsb-nzx1}{Phys. Rev. B {\bfseries{112}} (4), 045112 (2025)}}
\end{frame}
\end{comment}


\begin{frame}
	\frametitle{\insertsectionhead}
	\begin{figure}[htp!]	 
			\centering 
			\includegraphics[trim={0cm 0cm 0cm 0cm}, width = 1 \linewidth]{Illustrations/sketch} 			
	\end{figure}\blfootnote{B. Michen, T. Pokart, and J.C. Budich,
\href{https://journals.aps.org/prb/abstract/10.1103/dqsb-nzx1}{Phys. Rev. B {\bfseries{112}} (4), 045112 (2025)}}
\end{frame}

\begin{frame}
	\frametitle{\insertsectionhead}
	\begin{figure}[htp!]	 
			\centering 
			\includegraphics[trim={0cm 0cm 0cm 0.5cm}, width = 0.9 \linewidth]{Illustrations/compare_fidelity.pdf} 			
			\caption*{(a) MPSTE data for fidelity achieved with flux term. Linear scaling $\tau_\mathrm{tot} \propto L$.\\		
			 (b) Performance without flux hopping is qualitatively worse, quadratic scaling $\tau_\mathrm{tot} \propto L^2$.} 
	\end{figure}\blfootnote{B. Michen, T. Pokart, and J.C. Budich,
\href{https://journals.aps.org/prb/abstract/10.1103/dqsb-nzx1}{Phys. Rev. B {\bfseries{112}} (4), 045112 (2025)}}
\end{frame}

\section{Non-Linear Interference Challenging Topological Protection of Chiral Edge States}

\begin{frame}
	\frametitle{\insertsectionhead}
	\begin{columns}[onlytextwidth,T]
		\begin{column}{.45\textwidth}
			\textbf{\large Non-linear Floquet systems}	\\
			\vspace{0.5em}
	 Some interactions can be modeled as non-linear potential:
\begin{align}
i \frac{\mathrm d}{\mathrm d t} \psi_j(t) = \sum_{j'} H_F(j,j',t) \psi_{j'}(t) + \gamma |\psi_j(t)|^2 \psi_j(t). \nonumber
\end{align}
\begin{itemize}[<+(1)->]
\item Nonlinear term can destroy topological edge states of $H_F(j,j',t)$ through scattering
\item Relative phase acts as switch
\end{itemize}
		\end{column}
		
		\begin{column}{.45\textwidth}
						\flushright
			\begin{figure}[htp!]	 
				\includegraphics[trim={0cm 0cm 1cm 0cm}, width=  \linewidth]{Illustrations/Floquet_illustration.png} 
			\end{figure}
		\end{column}
	\end{columns}
\blfootnote{B. Michen and J.C. Budich,
\href{https://journals.aps.org/pra/abstract/10.1103/PhysRevA.108.L021501}{Phys. Rev. A {\bfseries{108}} (2), L021501 (2023)}}
\end{frame}

\begin{frame}
	\frametitle{\insertsectionhead}
	\begin{figure}[htp!]	 
			\centering 
			\includegraphics[trim={0cm 0.5cm 0cm 1cm}, width = 0.85 \linewidth]{Illustrations/Minmal_Floquet_model.png} 			
			\caption*{Floquet Hamiltonian in anomalous topological phase (realized recently in photonic crystals).} 
	\end{figure}\vspace{-1.5em} \blfootnote{B. Michen and J.C. Budich,
\href{https://journals.aps.org/pra/abstract/10.1103/PhysRevA.108.L021501}{Phys. Rev. A {\bfseries{108}} (2), L021501 (2023)}}
\end{frame}


\begin{frame}
	\frametitle{\insertsectionhead}
	\begin{columns}[onlytextwidth,T]
		\begin{column}{.55\textwidth}
			\textbf{\large Concrete approach}	\\
			\vspace{0.5em}
\begin{itemize}[<+(1)->]
\item Numerically solve non-linear dynamics as ODE with given initial conditions
\item Continuity equation yields current $\phi_{\bm \delta, \bm j}(t)$ through lattice link
\begin{align}
\frac{\mathrm d}{\mathrm d t} |\psi_{\bm j}|^2 = \sum_{\bm \delta \in \mathrm {NN}} \underbrace{2 \mathrm {Im} [J_{\bm \delta} (t) \psi^*_{\bm j} \psi_{\bm j + \bm \delta}]}_{-\phi_{\bm \delta, \bm j}(t)} \nonumber
\end{align}
\item Pick link and integrate current over time $\phi_\mathrm{T} = \int_0^{10T} \phi_{\bm j, \bm \delta}(t) \mathrm d t$ to measure scattering 
\end{itemize}
\end{column}
		
		\begin{column}{.55\textwidth}
						\flushright
			\begin{figure}[htp!]	 
				\includegraphics[trim={0cm 0cm 0cm 0cm}, width= 0.8 \linewidth]{Illustrations/phot_flux.png} 
			\end{figure}
		\end{column}
	\end{columns}
\blfootnote{B. Michen and J.C. Budich,
\href{https://journals.aps.org/pra/abstract/10.1103/PhysRevA.108.L021501}{Phys. Rev. A {\bfseries{108}} (2), L021501 (2023)}}
\end{frame}

\section{Quantum Hall Effect without Chern Bands}

\begin{frame}
\frametitle{\insertsectionhead}
\textbf{\large Hall conductance as a function of Fermi energy} \\
\vspace{0.5em}
For translation-invariant system with Bloch Hamiltonian $H(\bm k)$, eigenvectors $|u(\bm k) \rangle$, energies $E_{\bm k}$, and Berry curvature $\mathcal F = i [\langle \partial_{k_x} u| \partial_{k_y} u \rangle - \langle \partial_{k_y} u| \partial_{k_x} u \rangle]$, we have:
\begin{align}
\sigma_{xy}(E_\mathrm{F}) = \frac{e^2}{h} \frac{1}{2\pi}\int_{E_{\bm k}<E_\mathrm{F}}\mathrm{d}^2k \,  \mathcal F. \nonumber
\end{align}

\begin{itemize}[<+(1)->]
\item For $E_\mathrm{F}$ in band gap, $\sigma_{xy}$ is integer (Chern number)
\item For $E_\mathrm{F}$ in band spectrum, $\sigma_{xy}$ takes continuous values
\item Random disorder forces mobility gap and integer value of $\sigma_{xy}$ at any $E_\mathrm{F}$!
\end{itemize}
\end{frame}

\begin{frame}
\frametitle{\insertsectionhead}
\textbf{\large Setup for new physics in minimal two-band model} \\
\vspace{0.5em}
\begin{itemize}[<+(1)->]
\item Construct $\hat H_0$ with trivial ($\mathcal C = 0$) bands but $\sigma_{xy}(E_\mathrm{F}) \geq 0.5 e^2 / h$ in large $E_\mathrm{F}$ window
\item Add random disorder with amplitude $W$:
\begin{align*}
\hat H = \hat H_0 + \hat W.
\end{align*}
\item Probe transport properties using Kwant package
\end{itemize}
\vspace{0.5em}
\pause
$\Rightarrow$ Disorder stabilizes $\sigma_{xy} = e^2 / h$. IQHE in trivial band structure!\blfootnote{B. Michen and J. C. Budich,
\href{https://journals.aps.org/prl/abstract/10.1103/dgd8-4gzf}{Phys. Rev. Lett. {\bfseries{135}} (18), 186603 (2025)}}
\end{frame}

\begin{frame}
	\frametitle{\insertsectionhead}
	\begin{figure}[htp!]	 
			\centering 
			\includegraphics[trim={0cm 0cm 0cm 0.5cm}, width = 0.9 \linewidth]{Illustrations/QHE_wo_CB.png} 			
			\caption*{(a) Band structure of $\hat H_0$. (b) 4-terminal Hall conductance clean and with disorder.} 
	\end{figure}\blfootnote{B. Michen and J. C. Budich,
\href{https://journals.aps.org/prl/abstract/10.1103/dgd8-4gzf}{Phys. Rev. Lett. {\bfseries{135}} (18), 186603 (2025)}}
\vspace{-1em}
\end{frame}

\begin{frame}
	\frametitle{\insertsectionhead}

\begin{columns}[onlytextwidth,T]
\begin{column}{.45\textwidth}
\textbf{\large Finite size effects}	\\
\vspace{0.5em}	
	\begin{itemize}[<+(1)->]
		\item For $L \to \infty$, any $W > 0$ stabilizes large IQH window
		\item Key difference to topological Anderson insulator!
		\item $W \to 0$ means diverging localization length \\
		$\Rightarrow$ Order-of-limits problem between $W \to 0$ and $L \to \infty$
	\end{itemize}
\end{column}	
\begin{column}{.55\textwidth}
	\flushright
	\begin{figure}[htp!]	 
	\includegraphics[trim={-0.1cm 0cm 2.5cm 1cm}, width= 1 \linewidth]{Illustrations/transport_W_EF.png} 
	\caption*{Two-terminal transport along $x$-direction for system size $N_x = 600$, $N_y = 150$. (a) OBC along $y$, (b) PBC along $y$. Inset: Finite size scaling.}
	\end{figure}
\end{column}
\end{columns}
\blfootnote{B. Michen and J. C. Budich,
\href{https://journals.aps.org/prl/abstract/10.1103/dgd8-4gzf}{Phys. Rev. Lett. {\bfseries{135}} (18), 186603 (2025)}}
\end{frame}

\section{Future research}

\begin{frame}
\frametitle{\insertsectionhead}
\textbf{\large More on the QHE without Chern bands}
\pause
	\begin{itemize}[<+->]
		\item Tackle order-of-limits problem by calculating real-space GF for weak disorder
		\begin{align*}
		\left \langle \bm r \left | [E_\mathrm{F} - (\hat H_0 + \hat W) + i \eta]^{-1} \right | \bm r' \right \rangle \propto e^{-|\bm r- \bm r'| / \xi}
		\end{align*}
		\item Topological characterization of GFs (can we show that $\sigma_{xy} > 0.5 e^2 / h$ always becomes $\sigma_{xy} = e^2 / h$) for small $W$?
		\item Replica theory treatment (Pruisken's field theory)?
		\item Interplay with altermagnetic physics?
	\end{itemize}
\end{frame}

\begin{frame}
\frametitle{\insertsectionhead}
\textbf{\large Research on Geometric Floquet Theory}
\vspace{0.5em}	

Decomposition of periodic Hamiltonian 
\begin{align*}
H(t) = H_{\mathcal K}(t) + A_{\mathcal K}(t) 
\end{align*}
allows for unique ordering of Floquet states\footnote[1]{P. M. Schindler and M. Bukov, \emph{Geometric Floquet Theory},
\href{https://journals.aps.org/prx/abstract/10.1103/7l91-gw77}{PRX {\bfseries{15}} (3), 031037 (2025)}} according to average energies of $H_{\mathcal K}$.
\vspace{0.5em}	
\pause
	\begin{itemize}[<+->]
		\item Characterization of Floquet ground states (topological long-range entanglement and degeneracy?) 
		\item Further develop link between geometric Floquet theory and anomalous Floquet topological states
	\end{itemize}
\end{frame}

\begin{frame}
  \vfill
  \centering
  \begin{beamercolorbox}[sep=8pt,center,shadow=true,rounded=true]{title}
    \Large{Thank you for your attention!}
  \end{beamercolorbox}
  \vfill
\end{frame}


\end{document}

  