%%%%%%%%%%%%%%%%% PREAMBLE %%%%%%%%%%%%%%%%%%%%%%%%%%%%
\documentclass[letterpaper,11pt,oneside]{article}
\usepackage[utf8]{inputenc}
\usepackage{setspace}
\usepackage[colorlinks, citecolor = blue, linkcolor = red, urlcolor = blue]{hyperref}
\usepackage{comment}
\usepackage{multirow}
\usepackage{makecell}
\usepackage{xcolor}

\usepackage{graphicx}
\graphicspath{ {images/}}
\usepackage[left=1in, right=1in, bottom=1.25in, top=1.25in]{geometry}

\newcommand*{\Skype}{\href{skype:john.smith?add}{john.smith}}
\newcommand{\Absender}[1][\normalsize]{\Skype}
\newcommand{\fscv}{\normalsize}

\pagenumbering{gobble}

%%%%%%%%%%%%%%%%% END OF PREAMBLE %%%%%%%%%%%%%%%%%%%%%
\def\CC{{C\nolinebreak[4]\hspace{-.05em}\raisebox{.4ex}{\tiny\bf ++}}}

\begin{document}

\setlength\parindent{0cm}
\pagenumbering{gobble}

%%%%%%%%%%%%%%%%% NAME OF APPLICANT %%%%%%%%%%%%%%%%%%%

\noindent \LARGE{\textbf{Lebenslauf}} \\
\noindent \LARGE{\textbf{Benjamin Michen}} \\
\vspace{-2ex}
\hrule
\normalsize

%%%%%%%%%%%%%%%%% CONTACT INFORMATION %%%%%%%%%%%%%%%%%

\begin{center}
\begin{tabular}{r l r l}
Geburtsdatum: & 2.\ September 1996 & \hspace{0.25in} E-Mail: & \href{mailto:benjamin.michen@tu-dresden.de}{benjamin.michen@gmx.de} \\
Adresse: & Clara-Viebig-Str.~3, & \hspace{0.25in} Telefon: & +49 178 8723598 \\
& 01159 Dresden, Deutschland
\end{tabular}
\end{center}

\vspace{1em}

%%%%%%%%%%%%%%%%% MAIN BODY %%%%%%%%%%%%%%%%%%%%%%%%%%%

\noindent \begin{tabular}{@{} l l }

\fscv{Ausbildung} & \textbf{Abitur} mit der Gesamtnote 1,4 am J.-G.-Herder-Gymnasium Pirna, 2013--2015. \\
&\\
&\textbf{B.\,Sc.\ Physik} mit der Gesamtnote 1,6 an der TU Dresden, 2016--2019. \\
&\\
&\textbf{M.\,Sc.\ Physik} mit der Gesamtnote 1,0 an der TU Dresden, 2019--2021. \\
& \\
&\textbf{Promotion (Dr.\ rer.\ nat.)\ Physik} an der TU Dresden, 2021--2025, \\
&Dissertation: ``Nudging Quantum Matter over the Edge of Topological Order'' \\
&eingereicht im November 2025; derzeit im Begutachtungsverfahren, Verteidigung \\
&voraussichtlich Anfang 2026. \\
& \\
\fscv{Arbeits-} & \textbf{Arbeitsgruppe Quanten-Vielteilchentheorie, TU Dresden, 2020--2025} \\
\fscv{erfahrung} & Mathematische und numerische Modellierung der Physik wechselwirkender \\
& und topologischer Quantensysteme unter Prof.~Jan Carl Budich. Außerdem\\
 &verantwortlich für das Verfassen und Illustrieren wissenschaflticher Publiaktionen,\\
& siehe dazu auch mein \href{https://scholar.google.com/citations?user=9lyfMKsAAAAJ&hl=de}{Profil} bei Google Scholar.\\
&\\
& \textbf{Fraunhofer FEP Dresden, 2018--2021} \\
& Tätigkeit als studentischer Mitarbeiter für die Durchführung und \\
& Dokumentation von Elektronenstrahl-Experimenten sowie für numerische \\
&Simulationen elektromagnetischer Linsen mit finite-Elemente-Methoden. \\
&\\
\fscv{Lehr-} & \textbf{TU Dresden} \\
\fscv{erfahrung} & Übungsleiter/Tutor für zahlreiche Lehrveranstaltungen der Theoretischen Physik. \\
& \\

\fscv{Sprachen} & Deutsch (Muttersprache), Englisch (fließend), Spanisch (Niveau B1). \\
\fscv{und Fähigkeiten} & Programmiererfahrung in Python, Julia, \CC, und Mathematica. \\
& Erfahrung mit Numerik aller Art.\\
& Solide und breitgefächerte Kenntnisse in angewandter Mathematik und Topologie sowie Differenzialgeometrie\\

\\
\fscv{Sonstiges} & Work \& Travel in Neuseeland im Rahmen eines Gap Years, 2015--2016.\\


\end{tabular}

\clearpage

\end{document}